\section{Introduction}
\label{Intro:Introduction}
In nature, each living creature has specific characteristics that determine its role in its environment.  From microscopic entities to blue whales, all organisms possess patterns that enable them to live, reproduce and die. Whilst the life cycle is  a process common in all life forms, differences between them are clear. In the colder environments, living entities like polar bears with thick fur have different characteristics to animals in hotter environments as a camel's body temperature fluctuates with its surroundings. Arguably, the most significant difference is behaviour as it defines the role every existing organism has. Consider the example of a predator and its prey.  Their roles can be subjective and the prey in this instance can become a predator to another. That is the essence of behaviour; how particular it is to each type of organism but with a final common goal, survival.  
\\\\Humans are inherently complex and their basic rules of survival have changed. Survival appears to be secondary to superficial concerns, an example like eating has been belittled with the focal concern to survive. This peculiarity could be one of the main reasons for changes in human behaviour. The key word for me is rules. Rules dictate who we are and how we behave  around others, and while following these rules we adapt to prevail in the environment.  It is bold to say that we all play a game, life, and the rules are not only dictated by our natural environment but also by the social environment.
\subsection{Project objectives}
This project aims to give a general framework of a particular branch of game theory, evolutionary game theory, and to build a library in python to solve game theoretical problems presented in the normal form with considerations made from evolutionary game theory. The program built in Python will use a simplified genetic algorithm. The approach that will be used for modeling is agent based modeling (ABM), which will be introduced further in this work. Basic relevant knowledge about evolutionary game theory will be presented. Python programming language will be used to build a library capable of solving some well known normal form games (games is strategic form) like prisoners' dilemma, matching pennies, battle of sexes and some others. A Python package will be used to simulate an experiment made by the professor in political science Robert Axelrod in 1980, who's goal was to observe how cooperation evolved in an environment where selfish behaviours existed when interacting through time. 
\subsection{Project structure}
This project has the following structure. In the next section which is literature review, will have some background about game theory, which will include normal form games and Nash equilibrium; followed by an introduction to evolutionary game theory, a definition for evolutionary stable strategy and how it relates to Nash equilibrium; background information about genetic algorithm and how it relates to evolutionary game theory; background information about agent-based modeling and the object oriented programming style; finally the considerations made to use a genetic type algorithm in the project. Section three will explain the use of the approach test driven development for building code; an introduction to version control system Git for creating packages and documents in which others can contribute; general guidelines of what tools and webpage will be used for providing potential users the instructions and relevant information about the project. In section four an explanation of the modules contained in the library, the interaction between them, and the variables and methods contained within the modules will be given. Finally for section five, examples of the different outputs the python library provides will be presented with some different combinations of parameters that can be used, applied to some well known game theoretical games in the normal form. During this project the words library and package, when refering to code built `ablearn', will be used interchangeably. 
