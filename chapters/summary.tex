An overvie of `evolutionary game theory' is given in this project. This concept is relevant for understanding, how in a biological context, game theory can be applicable to concepts of survival of species. Knowing evolutionary game theory helps to understand better the concept of the ``genetic algorithm'' which was used for creating the python library for this project. The ``genetic algorithm'' used was simple, and it mainly uses concepts and assumptions from ``evolutionary game theory''. The algorithm consists in the creation of agents with characteristics. For defining agents, the approach was agent-based modelling, which main purpose is to simulate how the interactions between different agents has an effect on the individual agent and in the whole system. The main characteristics for each agent are: each created agent possesses only one strategy which it will use until it is eliminated, and by using this strategy the agent is capable of accumulating a utility. When the utility is high it will help it to endure during the simulation, and it will very likely be able to produce a copy of itself, consequently the strategy it passes on will be present in more agents during the simulation. On the other hand, agents with a low accumulated utility will not completely disappear, because of the mutation rate, but it will be present in a low proportion along the simulation. Parameters that affected  the simulation were explained, such as death rate, mutation rate, exploitation rate and initial distribution of strategies. For being able to build a simulation with all of the features mentioned ``Python'' was used which is an object oriented programming language. Once a Python library capable of running simulations was created some of the most known games in game theory like prisoner's dilemma, matching pennies, battle of sexes and hawk-dove (in evolutionary game theory) were simulated. All these games were tested with different combination of parameters, to observe their individual behaviour. And peculiarities for each game, according to their types of equilibria (mixed or pure) were observed. After these games were tested an additional game was tested. This game was produced with a Python package called ``axelrod'', which allows simulate tournaments with predetermined strategies. From ``axelrod'' a payoff matrix was taken, this payoff matrix was transformed into a bimatrix and introduced into the Python library created for this project. The ``axelrod'' package also produces a stack plot, the library for this project also produces a stackplot and both stack plots were compared. After producing the different outputs for the different games and also after comparing results with the results from ``axelrod'' package, the library built for this project which name is ``ablearn'' did produce expected results for the known games for game theory and produced somehow similar stack plots to the stack plots produced by the ``axelrod'' package.