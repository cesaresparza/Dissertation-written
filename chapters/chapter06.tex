\label{ch:conclusions}

\section{General conclusions about the results}
Game theory has proven to give relevant results for many fields for a long period of time now, even if its application nowadays is mostly in economics. Evolutionary game theory which was derived from game theory, even if less complex assumptions can give equally good and accurate results, and nowadays has had more acceptance  and its application has grown too. This suggests  complex questions can be responded by answers where simpler criteria is assumed.  Even if `ablearn' package at this stage can only solve a limited type of games, it has proven to give reliable results.   The algorithm used in `ablearn' is simple, but given the number of iterations it would be long and not a very practical method for calculating manually. With all the available tools and advancement in technology this long process is made in less time and with less complications. 
From the analysis of the simulations' outputs it can be noted that death rate had a big influence. A low death rate allows agents with certain strategy (most commonly the dominant strategy) to have a higher presence in the population  and when the rate is high agent with certain strategy (most commonly dominant strategy) still have a higher presence in the population but it is not as significant since in every generation a big number is eliminated and the agents with dominant strategies may not always be the only ones able to be reproduced. This difference in which agents with dominant strategies may not always be the ones to be reproduced, highly depends on the other two remaining rates (mutation rate and exploitation rate). As it can be seen between mutation and exploitation rate, the first has a higher impact in the outcome. One of the reasons is because of the costraint it adds to the dominant strategy agents (not allowing them to reproduce), another reason is that mutation rate acts as a second criteria so even after an agent has been selected to reproduce by the exploitation rate it is still possible that the mutation rate changes the selected agent and one with a different strategy may be chosen.  With a high mutation rate even a possible overwhelming dominant strategy will not have an absolute presence in the population, although it will still be dominant. As the death rate increases in combination with a high mutation rate, the distribution of the agents with different strategies will tend to be equal for all in proportions. Exploitation rate is not very relevant when it is in combination with a high mutation rate, but with a low mutation rate the effect from it can be noticed. But combined with a low mutation rate and a relatively high death rate, it can give equal opportunity to any agent with any strategy to reproduce if the exploitation rate is low. Another important thing that could be observed how results could be affected by adding an initial strategy distribution into a game where 2 pure Nash equilibria and a mixed Nash equilibria exist. It could be observed in the graphs and the stack plots that when giving a higher value in the initial distribution to one of the strategies, that strategy became the dominant strategy.
When comparing the resulting stack plots from `ablearn' package's with the resulting stack plot from `axelrod' pacakge's, it could be seen  that two different techniques were being used to estimate the behaviour of the strategies, but the results only slightly different. In a first view the stack plot from `axelrod' seems smoother that is because it uses a probabilistic method to calculate the values, on the other hand `ablearn' does all the iterations, the stochasticity produced is given the random interactions and the randomness that mutation and exploitation rate add to the reproduction of the strategies.

\subsection{Further work}

The package was created with the intention that contributions, from others in the field interested in the package, can be made to enhance it. Whilst in this project a code was built for creating agent based models that will use learning algorithms to solve games, the PYTHON functionalities are vast and possibilities for further developments are endless. One of the advantages of using a VCS, in this project Git is used,  is that contributors can continually update work; in addition the tools for comparing contributions, checking for possible conflicts within the code and the communication offered by websites like Github make contributing package development easier. This thriving community provides opportunities for distributing packages and getting acknolowledgement to attract other enthusiasts and encourage contribution. This project is a contirbution to this community and with hopes in the future I along with others will improve it.

The results that `ablearn' package in it's current state provides, are reliable for users to see how strategies in symmetric normal form games perform through time. It helps to identify in a graphic form Nash equilibria and evolutionary stable strategies. With the different combinations that can be used as input (number of generations, number of rounds, death rate, mutation rate, exploitation rate and distribution) it is able to produce the results of various different scenarios and should help the user how under different conditions the behaviour of ,dominating or dominated, strategies changes.

There are many possibilities for making the code aplicable to the variety of existing type of games. Some of the examples include developing the package so that it can solve a variety of games with n-players, with other types of environments and interactions i.e. instead of agents interacting randomly with any other agent, allow them to interact only with a certain number of agents which can create a network environment. Another possible enhancement includes incorporating the option of analysing historical data from a certain type of agents and generating a prediction of an outcome i.e. taking the records from the time runners in the same category and that run the same distance in races under different conditions such as weather, place and other possible relevant variables, and be able to generate a prediction of which runner would be more likely to win a race against the other runners under certain conditions. 
The main purpose is to explore as many available algorithms to solve as many types of games as possible. Also it would be desirable that after a person reads this project an interest is drawn towards algorithms that can be applied in game theory and python's language broad potential applications, 

