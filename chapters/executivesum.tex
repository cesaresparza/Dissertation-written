Game theory is field that has been used more frequently nowadays in a wider range of fields. The expansion in fields of application, has led to variations in the original assumptions used for games in game theory. In this project a review of  ``evolutionary game theory'' was presented, which is a subfield of game theory. Evolutionary game theory is the application of game theory to the concept of evolution. This was relevant for understanding, how seen from a biological sense, game theory can be applicable to concepts of survival of species. With this as a basic idea it is easier to understand the concept of the ``genetic algorithm'' used to create a python library for this project. The ``genetic algorithm'' used was simple, and it mainly uses concepts and assumptions from ``evolutionary game theory''. The algorithm basically consists in the creation of agents with characteristics. The approach taken for defining agents, was agent-based modelling, which main purpose is to simulate the how the interactions between different agents has an effect on the individual agent and in the whole system. The main characteristics for each agent are: each created agent has only one strategy which it will use until it is eliminated and by using this strategy the agent is capable of accumulating a utility. When the utility is high it will help it to endure during the simulation, and it will very likely be able to produce a copy of itself, consequently the strategy it passes on will be present in more agents during the simulation. On the other hand, agents with a low accumulated utility will probably disappear as the simulation progresses. Parameters that affected  the simulation were introduced, such as death rate, mutation rate, exploitation rate and initial distribution of strategies. For being able to build a simulation with all of the features mentioned ``Python'' was used which an object oriented programming language. Once a Python library capable of running simulations was created some of the most known games in game theory like prisoner's dilemma, matching pennies, battle of sexes and hawk-dove (in evolutionary game theory) were simulated. All these games were tested with different combination of parameters, to observe their individual behaviour. And peculiarities for each game, according to their types of equilibria (mixed or pure) were observed. It could be observed in games like `prisoner's dilemma', that when a pure Nash equilibrium exists, that strategy will be the dominating strategy for all agents. In games like `matching pennies', where only a mixed strategy Equilibrium exists, it could be observed that at different points in time a different strategy dominates; this reflects how the mixed Nash equilibrium (1/2, 1/2) behaves. Games where there exist two pure Nash equilibria and a mixed Nash equilibria like `battle of sexes', show that given the inestability that the 2 Nash equilibria produce on the mixed strategy equilibrium, the dominating strategy will change when the proportions of the total population favour one of the two pure strategies. Finally, for games like `hawk-dove' where either one of the agents can obtain a higher utility when a non-symmetric strategy and at the same time one of the two symmetric strategies provide more utility, the result is that the agents will choose one of the two non-symmetric strategies even if it gives a higher payoff to one of them. After analysing these games an additional was tested. This game was produced with the Python package ``axelrod'', and allows simulate tournaments with predetermined strategies. From ``axelrod'' a payoff matrix was taken, this payoff matrix was transformed into a bimatrix and introduced into the Python library created for this project to produce a stack plot with `ablearn' library. The stack plots produced by both packages were compared, and it could be seen that produced a similar output. From these outputs the characteristic that was expected to be present in stack plots from `ablearn' was that, as in the stack plots produced by the `axelrod' package, the strategies considered `Cooperative' dominated the population and prevailed through time.